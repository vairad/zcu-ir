\documentclass[12pt, a4paper]{article}
\usepackage[utf8]{inputenc}
\usepackage[IL2]{fontenc}
\usepackage[czech]{babel}
\usepackage{graphicx}

\begin{document}
\begin{figure}[h!]
\centering
\includegraphics[bb= 0 0 820 445 , width=75mm]{favlogo.jpg}
\end{figure}

{\centering
{\huge Systém automatické indexace}\\[1em]
{\large KIV/IR}\\[11,5cm]
}

\begin{tabular}{l r}
student: & Radek VAIS\\
os. číslo: & A17N0093P\\
mail: & vaisr@students.zcu.cz\\
datum: & 20.5.2018\\
\end{tabular}

\thispagestyle{empty}
\newpage

%========================================
%========================================
%========================================
%========================================
%========================================
\section{Zadání} %=====================================================================================================

S využitím připravených rozhraní v jazyce Java navrhněte a implementujte systém automatické indexace a vyhledávání dokumentů.

\subsection{Minimální funkčnost:}

Tokenizace, Preprocessing (stopwords remover, stemmer/lemmatizer), vytvoření in-memory invertovaného indexu, tf-idf model, cosine similarity,  vyhledávání pomocí dotazu vrací top x výsledků seřazených dle relevance, vyhledávání s pomocí logických operátorů AND, OR, NOT, podrobná dokumentace (programátorská i uživatelská)

\subsection{Nadstandardní funkčnost:}

File-based index (1b), pozdější doindexování dat (1b), ošetření např. HTML tagů (1b), detekce jazyka (1b), vylepšení vyhledávání (1b), vyhledávání frází (i stop slova)(1b), vyhledávání v okolí slova (1b), více scoring modelů (1b),  indexování webového obsahu (2b), další předzpracování normalizace (1b), webové rozhraní/GUI (2b), napovídání keywords (1b), podpora více polí pro dokument (např. datum)(1b), CRUD indexovaných dokumentů (2b), zvýraznění hledaného textu v náhledu výsledků (1b), dokumentace psaná v TEXu (1b), atd. (xb)


%====================================================================================================================================================
\newpage

\section{Popis řešení}


\section{Ovládání}

\subsection{Sestavení}


\subsection{Spuštění}


\section{Závěr}

V rámci této semestrální práce byl vytvořen systém automatické indexace a vyhledávání dokumentů s následujícími vlastnostmi.

\begin{itemize}
\item Předzpracování
	\begin{itemize}
		\item Odstranění diakritiky
		\item Stemming slov
		\item Přeskočení stop slov.
		\item Tokenizace pomocí regulárních výrazů.
	\end{itemize}
\item{Indexace}
		\begin{itemize}
		\item Vytváření in-memory invertovaného indexu.
		\item Načítání indexu do paměti z uloženého souboru.
	\end{itemize}
\item{Vyhledávání}
	\begin{itemize}
		\item Vyhledávání pomocí booleovských operátorů.
		\item TF-IDF model.
		\item Cosinová podobnost.
	\end{itemize}
\item{Uživatelské rozhraní}
	\begin{itemize}
		\item Implementováno pomocí prostředků JavaFX
		\item Možnost zadat dotaz.
		\item Zobrazení top 10 výsledků.
		\item Možnost načtení uloženého indexu.
	\end{itemize}
\item{Dokumentace}
	\begin{itemize}
		\item Akademická dokumentace sázena systémem \LaTeX
		\item Uživatelská dokumentace. (součást akademické)
		\item Programátorská dokumentace JavaDoc. (součást zdrojových souborů) 
	\end{itemize}	
\end{itemize}

\end{document}